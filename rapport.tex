\documentclass{article}
\usepackage{amsmath}
\usepackage{amsfonts}
\usepackage{amssymb}
\usepackage{graphicx}
\usepackage{array}
\usepackage{booktabs}
\usepackage{xcolor,colortbl}

\usepackage[a4paper, margin=1in]{geometry} % Adjust the margin as needed

\title{Rapport de Projet d'analyse d'image\\ Détection et classification de Balise Cardinale}
\author{BACK Raphaël, BOUTET Paul}
\date{Janvier 2025}

\begin{document}

\maketitle

\begin{abstract}
    Ce rapport présente notre projet d'analyse d'image visant à détecter et classifier les balises cardinales. Nous avons utilisé plusieurs méthodes pour segmenter les balises, les classifier par couleur et par forme, et mis en place une double validation pour améliorer la précision. Enfin, nous proposons une méthode pour déterminer la direction de navigation à partir des balises détectées. Nous présentons les résultats obtenus et discutons des améliorations possibles.
\end{abstract}

\setcounter{tocdepth}{2}
\tableofcontents

\newpage

\section{Segmentation des balises}

L'objectif est de segmenter les balises cardinales dans une image pour obtenir
un masque ne contenant que les balises. Nous utilisons plusieurs
reconstructions par dilatation.

\subsection{Masque de couleur jaune}

Nous obtenons le masque de couleur jaune qui servira de marqueur pour la suite
de la segmentation.

\subsubsection{Conversion en HSV et seuillage}
Pour obtenir des marqueurs internes aux balises cardinales, nous segmentons les
zones de couleur jaune en convertissant l'image en HSV et en procédant à un
seuillage. La composante H isole les zones jaunes et un seuillage sur la
composante S limite la détection de bancs de sable.

\subsubsection{Nettoyage}
Nous effectuons un nettoyage par ouvertures morphologiques pour éliminer les
petites zones de bruit et les zones touchant les bords de l'image.

\subsection{Seuillage du gradient sur la composante bleue}

Nous obtenons un masque de l'image contenant les balises cardinales pour y
appliquer une reconstruction par dilatation.

\subsubsection{Calcul du gradient}
Pour détecter les contours des balises, nous calculons un gradient sur la
composante bleue de l'image. Cette méthode met en évidence les contours des
balises noires et jaunes, créant un fort contraste avec l'arrière-plan bleu.

Nous appliquons un seuillage sur le gradient obtenu pour obtenir le masque
voulu.

\subsubsection{Troncature du gradient}
Pour supprimer les grandes zones de plages en fond et limiter les grands objets
en arrière-plan, nous limitons le calcul du gradient à une zone de l'image
délimitée par le masque jaune obtenu précédemment.

\subsection{Reconstruction par dilatation}

Nous utilisons les marqueurs obtenus pour effectuer une reconstruction par
dilatation sur le gradient de la composante bleue, segmentant ainsi les balises
cardinales.

Malheureusement, nous détectons tout objet contenant une zone jaune. Des petits
objets tels que des bouées ou des bateaux peuvent être détectés. Nous devons
donc effectuer une étape de post-traitement pour ne garder que les balises
cardinales.

\subsection{Suppression des objets non pertinents}

Nous réappliquons la méthode précédente (reconstruction par dilatation) pour ne
garder que les balises cardinales.

Nous calculons un nouveau masque qui servira de marqueur à la future
reconstruction par dilatation.

\subsubsection{Détection des zones jaunes}
Nous réappliquons la méthode de seuillage sur la composante jaune de l'image
pour obtenir les zones jaunes dans les balises. Nous appliquons notre calcul
des zones jaunes sur l'image masquée par le dernier masque obtenu.

\subsubsection{Suppression des objets trop petits}
Nous supprimons les objets trop petits pour être des balises cardinales en
calculant l'aire de chaque objet présent et en les filtrant en fonction de leur
aire en proportion de l'aire du plus grand objet.

\subsubsection{Suppression des objets lointains}
Après plusieurs tests, nous avons remarqué que certaines bouées ou objets assez
grands sont détectés. Nous décidons de supprimer les objets trop éloignés en
abscisse du plus grand objet détecté, considéré comme la partie jaune de la
balise cardinale.

\subsubsection{Reconstruction par dilatation}

Nous appliquons une reconstruction par dilatation sur le premier masque
reconstruit en utilisant le masque jaune nettoyé.

\subsection{Amélioration de la détection des triangles}

Les tests montrent que la détection des triangles n'est pas parfaite. Les
triangles sont souvent ratés car ils ne forment pas une composante connexe avec
les balises sur le gradient calculé.

Pour pallier ce problème, nous faisons une dilatation vers le haut de la balise
détectée et refaisons une reconstruction par dilatation sur le gradient de la
composante bleue pour détecter les triangles potentiellement oubliés.

Nous utilisons le masque final de l'étape 4 dilaté vers le haut comme marqueur
pour la reconstruction par dilatation.

Les résultats de cette méthode sont satisfaisants et permettent de détecter
plus de triangles.

\section{Classification par couleur}

Nous avons maintenant un masque de l'image ne contenant que la balise
classifier. L'objectif est de classifier la balise en fonction de sa couleur en
analysant la position des zones jaunes.

\subsection{Boîte englobante de la balise}

Nous calculons la boîte englobante de la balise pour déterminer la position des
zones jaunes par rapport à la balise.

En considérant que le masque ne comprend que la balise, nous calculons la boîte
englobante de la balise en utilisant la boîte englobante du masque.

La boîte englobante d'intérêt est celle de la balise sans les triangles. Pour
isoler cette zone, nous supprimons les 22 plus hauts pourcents de la boîte
englobante de la balise.

\subsection{Détection des boîtes englobantes des zones jaunes}

Pour calculer la position des zones jaunes par rapport à la balise, nous avons
besoin de leurs boîtes englobantes.

Nous détectons les zones jaunes par seuillage en HSV et fusionnons les bounding
boxes des zones jaunes trop proches ou se chevauchant.

Une étape de nettoyage est effectuée pour supprimer les zones jaunes trop
petites, correspondant à des bruits ou des ombres sur le pied de la balise.

\subsection{Classification}

Nous classifions la balise en fonction de la position des zones jaunes par
rapport à la balise.

Un premier filtre est effectué si 2 zones jaunes sont détectées, car seules les
balises cardinales Ouest ont 2 zones jaunes.

Nous calculons ensuite si la zone jaune restante est présente dans la partie
haute, médiane ou basse de la boîte englobante de la balise, déterminant ainsi
si la balise est Nord, Est ou Sud.

\section{Classification par triangles}

Nous repartons du masque binaire de l’image, ne contenant que la balise
classifier. Ce masque est une image en noir et blanc. L’objectif est de
classifier la balise en fonction de l’orientation des triangles en analysant
leur position.

\subsection{Boîte englobante des triangles}

En partant de la Bounding Box ne contenant que la balise, nous prenons la
partie supérieure contenant les triangles. Ils représentent environ 22\% de sa
hauteur, facilitant l’extraction de cette partie de l’image. Nous obtenons
alors uniquement les 2 triangles de la balise étudiée.

\subsection{Nettoyage des triangles}

Pour simplifier l’étape suivante et garantir des triangles bien définis, un
léger nettoyage est effectué par une opération morphologique d’ouverture avec
un disque comme élément structurant. Ce traitement élimine les éventuels bruits
autour des triangles et la barre qui les supporte.

\subsection{Classification}

L’image contenant les quatre triangles est divisée en quatre sous-images de
taille égale, correspondant chacune à un demi-triangle. Pour identifier la
balise, nous comptons le nombre de pixels blancs dans chaque demi-triangle.
Selon la distribution des pixels, nous déterminons le type de balise.

Exemple : Si la première image a plus de pixels que la deuxième, alors le
premier triangle correspond à une balise Sud ou Ouest. De même, si la troisième
image en a plus que la quatrième, alors c’est une balise Sud. Ce qui nous donne
un résultat pour la classification par triangles.

\section{Double Validation}

Pour améliorer la précision de notre algorithme, nous mettons en place une
double validation. L'idée est de vérifier que la classification par couleur et
par forme sont cohérentes. Nous comparons les résultats obtenus par les deux
méthodes et ne conservons que les balises classées de la même manière par les
deux méthodes.

\subsection{Paramètre de cohérence}
Nous créons un paramètre de confiance nommé cohérence (consistency) qui est un
booléen TRUE quand les 2 algorithmes sont cohérents et FALSE sinon.

La direction de navigation conseillée, explicitée dans la partie suivante, est
calculée seulement dans les cas cohérents.

\section{Recherche d'une direction de navigation}

La recherche de la direction de navigation se fait par simple calcul de la
position des balises par rapport à la position du bateau.

Nous affichons la direction de navigation conseillée en fonction de la position
des balises cardinales détectées directement sur l'image demandée.

\section{Analyse des résultats}

Les résultats suivants sont les résultats de classification des balises
cardinales sur les images de test. Ces résultats et statistiques sont calculés
sur le jeu de données de test de l'énoncé. Ci-dessous, le tableau des
résultats.

\begin{table}[h!]
    \centering
    \resizebox{\textwidth}{!}{%
        \begin{tabular}{|c|p{4cm}|c|c|c|}
            \hline
            \rowcolor{gray!60}
            \textbf{Image}       & \textbf{Segmentation Result} & \textbf{Color Classification} & \textbf{Triangle Classification} & \textbf{Consistency} \\ \hline

            \textbf{60dsP35.jpg} & Background kept              & West                          & North                            & FALSE                \\ \hline
            \rowcolor{gray!10}
            \textbf{7yVN7Wh.jpg} &                              & East                          & South                            & FALSE                \\ \hline
            \rowcolor{gray!0}
            \textbf{86LjH8X.jpg} & Water under kept             & South                         & South                            & TRUE                 \\ \hline
            \rowcolor{gray!10}
            \textbf{8K2kvSA.jpg} & Water under kept             & South                         & South                            & TRUE                 \\ \hline
            \rowcolor{gray!0}
            \textbf{A6hVLZ7.jpg} &                              & East                          & North                            & FALSE                \\ \hline
            \rowcolor{gray!10}
            \textbf{J3JGeTY.jpg} & Lack 1 triangle              & South                         & West                             & FALSE                \\ \hline
            \rowcolor{gray!0}
            \textbf{O1XtIg5.jpg} & Water under kept             & South                         & South                            & TRUE                 \\ \hline
            \rowcolor{gray!10}
            \textbf{YU750z5.jpg} & A bit of land behind         & South                         & South                            & TRUE                 \\ \hline
            \rowcolor{gray!0}
            \textbf{h3YP60d.jpg} & Water around kept            & West                          & North                            & FALSE                \\ \hline
            \rowcolor{gray!10}
            \textbf{i4R1U6v.jpg} & Half, lack triangles         & East                          & South                            & FALSE                \\ \hline
            \rowcolor{gray!40}
            \textbf{South}       &                              & 50\%                          & 60\%                             & 40\%                 \\ \hline
            \rowcolor{gray!0}
            \textbf{3DTzEGh.jpg} & Water around kept            & East                          & South                            & FALSE                \\ \hline
            \rowcolor{gray!10}
            \textbf{7F08T3Y.jpg} &                              & East                          & East                             & TRUE                 \\ \hline
            \rowcolor{gray!0}
            \textbf{8b3J5Me.jpg} & Lack triangles               & East                          & North                            & FALSE                \\ \hline
            \rowcolor{gray!10}
            \textbf{G3amY5r.jpg} & Triangles blurred            & East                          & East                             & TRUE                 \\ \hline
            \rowcolor{gray!0}
            \textbf{MI8n706.jpg} & Lack 1 triangle              & East                          & West                             & FALSE                \\ \hline
            \rowcolor{gray!10}
            \textbf{QO80o8u.jpg} &                              & East                          & East                             & TRUE                 \\ \hline
            \rowcolor{gray!0}
            \textbf{SHmk487.jpg} &                              & East                          & East                             & TRUE                 \\ \hline
            \rowcolor{gray!10}
            \textbf{UJNcds9.jpg} &                              & East                          & East                             & TRUE                 \\ \hline
            \rowcolor{gray!0}
            \textbf{d5KA2X6.jpg} & A bit of land behind         & East                          & East                             & TRUE                 \\ \hline
            \rowcolor{gray!10}
            \textbf{tA2n98e.jpg} & Cut in half horizontally     & North                         & East                             & FALSE                \\ \hline
            \rowcolor{gray!40}
            \textbf{East}        &                              & 90\%                          & 70\%                             & 60\%                 \\ \hline
            \rowcolor{gray!0}
            \textbf{3bFnQfx.jpg} &                              & North                         & North                            & TRUE                 \\ \hline
            \rowcolor{gray!10}
            \textbf{3zmY4P1.jpg} & Water under kept             & East                          & North                            & FALSE                \\ \hline
            \rowcolor{gray!0}
            \textbf{5I878xn.jpg} & Lack 1 triangle              & East                          & East                             & TRUE                 \\ \hline
            \rowcolor{gray!10}
            \textbf{5epK341.jpg} & Holes in triangles           & North                         & North                            & TRUE                 \\ \hline
            \rowcolor{gray!0}
            \textbf{863P5k3.jpg} & Land behind                  & North                         & North                            & TRUE                 \\ \hline
            \rowcolor{gray!10}
            \textbf{Pn1PEIM.jpg} &                              & North                         & North                            & TRUE                 \\ \hline
            \rowcolor{gray!0}
            \textbf{TgfIcDT.jpg} &                              & North                         & North                            & TRUE                 \\ \hline
            \rowcolor{gray!10}
            \textbf{Y79Jt8c.jpg} & Land behind                  & North                         & North                            & TRUE                 \\ \hline
            \rowcolor{gray!0}
            \textbf{bsO98Tk.jpg} & A bit of land behind         & East                          & North                            & FALSE                \\ \hline
            \rowcolor{gray!10}
            \textbf{fVrzdzC.jpg} & Triangles cut                & North                         & West                             & FALSE                \\ \hline
            \rowcolor{gray!0}
            \textbf{oeSph7h.jpg} & A bit of water under kept    & North                         & North                            & TRUE                 \\ \hline
            \rowcolor{gray!40}
            \textbf{North}       &                              & 72,73\%                       & 81,82\%                          & 72,73\%              \\ \hline
            \rowcolor{gray!0}
            \textbf{01Tn6V3.jpg} &                              & West                          & West                             & TRUE                 \\ \hline
            \rowcolor{gray!10}
            \textbf{08hYdUe.jpg} &                              & West                          & West                             & TRUE                 \\ \hline
            \rowcolor{gray!0}
            \textbf{63JKJVG.jpg} & Water under kept             & West                          & West                             & TRUE                 \\ \hline
            \rowcolor{gray!10}
            \textbf{65MsTb3.jpg} &                              & West                          & West                             & TRUE                 \\ \hline
            \rowcolor{gray!0}
            \textbf{8KzVm58.jpg} & Lack triangles               & South                         & West                             & FALSE                \\ \hline
            \rowcolor{gray!10}
            \textbf{Ao1UT0B.jpg} &                              & West                          & West                             & TRUE                 \\ \hline
            \rowcolor{gray!0}
            \textbf{IY2fAd3.jpg} &                              & West                          & West                             & TRUE                 \\ \hline
            \rowcolor{gray!10}
            \textbf{QI7MH37.jpg} & Water under kept             & West                          & West                             & TRUE                 \\ \hline
            \rowcolor{gray!0}
            \textbf{XP96z9K.jpg} & Lack triangles               & West                          & East                             & FALSE                \\ \hline
            \rowcolor{gray!10}
            \textbf{mtjUaK0.jpg} & Lack triangles               & West                          & North                            & FALSE                \\ \hline
            \rowcolor{gray!40}
            \textbf{West}        &                              & 90\%                          & 80\%                             & 70\%                 \\ \hline
            \rowcolor{gray!40}
            \textbf{Total}       &                              & \textbf{75,61\%}              & \textbf{73,17\%}                 & \textbf{60,97\%}     \\ \hline
        \end{tabular}%}
    }
    \caption{Résultats de classification des balises cardinales}\label{tab:segmentation_results}
\end{table}

\subsection{Segmentation}

On remarque que la segmentation des balises n'est pas parfaite. Il manque
parfois des triangles ou des parties de balises. Cela peut être dû à la qualité
de l'image, à la présence d'ombres ou à d'autres objets dans l'image qui
interfèrent avec la détection des balises.

Les problèmes les plus courants sont dus au fort gradient dans les vagues ou à
des objets en fond qui sont confondus avec la balise dans le gradient de
l'image.

On perd également parfois des triangles malgré notre méthode d'agrandissement
puis de reconstruction par dilatation.

\subsection{Classification par couleur}

La classification par couleur est relativement précise. Nous remarquons que
dans les cas ou la segmentation est bonne ou que l'on detecte un objet en
arrière plan, la classification par couleur est correcte. Cette méthode semble
donc globalement efficace pour classer les balises cardinales.

\subsection{Classification par triangles}

La classification par triangles semble moins efficace que la classification par
couleurs au vu des résultats généraux mais elle est très fortement dépendante
de la segmentation des balises. En effet, si la segmentation est mauvaise, il
est difficile de détecter les triangles.

On remarque en effet que la plupart des erreurs de cette méthode sont dues à
une mauvaise segmentation des balises plutôt qu'à une mauvaise classification
des triangles.

\section{Conclusion}

En conclusion, notre algorithme est relativement efficace pour classifier les
balises cardinales. La classification par couleur est plus précise que la
classification par triangle car plus robuste aux mauvaises segmentations. La
double validation permet d'améliorer la précision de notre algorithme.
Cependant, la segmentation des balises reste perfectible. Il serait intéressant
d'explorer d'autres méthodes de segmentation pour améliorer les résultats. Il
serait également intéressant de tester notre algorithme sur un plus grand jeu
de données pour évaluer sa robustesse.

Il pourrait également être intéressant d'explorer la piste d'une detection des
blocs noirs pour améliorer les résultats de la segmentation par couleur dans
les cas ou les balises ont une zone jaune grande en comparaison de la zone
noire.

La méthode par détection de triangles pourrait également être améliorée en
utilisant des méthodes de détection des contours plus précises et en appliquant
peut être une seconde segmentation lors de cette étape.

\section{Annexes}

Les exemples ci-dessous sont calculés sur l'image "5epK341.jpg"
\begin{figure}[h!]
    \centering
    \begin{minipage}{0.45\textwidth}
        \centering
        \includegraphics[width=\textwidth]{masquage.eps}
        \caption{Masquage des balises dans l'image.}
    \end{minipage}\hfill
    \begin{minipage}{0.45\textwidth}
        \centering
        \includegraphics[width=\textwidth]{colorEsti.eps}
        \caption{Estimation de la couleur des balises.}
    \end{minipage}

\end{figure}
\begin{figure}[h!]
    \centering\begin{minipage}{0.45\textwidth}
        \centering
        \includegraphics[width=\textwidth]{trianglesBB.eps}
        \caption{Bounding Box des triangles détectés.}
    \end{minipage}
    \begin{minipage}{0.45\textwidth}
        \centering
        \includegraphics[width=\textwidth]{triangles.eps}
        \caption{Triangles détectés.}
    \end{minipage}\hfill

\end{figure}

\end{document}