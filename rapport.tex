\documentclass{article}
\usepackage{amsmath}
\usepackage{amsfonts}
\usepackage{amssymb}
\usepackage{graphicx}
\usepackage{array}
\usepackage{booktabs}
\usepackage{xcolor,colortbl}

\usepackage[a4paper, margin=1in]{geometry} % Adjust the margin as needed

\title{Rapport de Projet d'analyse d'image\\ Détection et classification de Balise Cardinale}
\author{BACK Raphaël, BOUTET Paul}
\date{Janvier 2025}

\begin{document}

\maketitle

\begin{abstract}
    Ce rapport présente les différentes étapes de notre projet d'analyse d'image. L'objectif de ce projet est de détecter et classifier les balises cardinales dans une image. Nous avons utilisé plusieurs méthodes pour segmenter les balises, les classifier par couleur et par forme. Nous avons également mis en place une double validation pour améliorer la précision de notre algorithme. Enfin, nous avons proposé une méthode pour déterminer la direction de navigation à partir des balises détectées. Nous présentons les résultats obtenus et discutons des améliorations possibles.
\end{abstract}

\setcounter{tocdepth}{2}
\tableofcontents

\newpage

\section{Segmentation des balises}

L'objectif de cette partie est de segmenter les balises cardinales dans une
image, l'idée est d'obtenir un masque de l'image qui ne contient que les
balises cardinales. Nous allons ici utiliser plusieurs reconstruction par
dilatation sur notre image.

\subsection{Masque de couleur jaune}

Nous voulons ici obtenir le masque de couleur jaune qui nous servira de
marqueur pour la suite de la segmentation.

\subsubsection{Conversion en HSV et seuillage}
Pour avoir des premiers marqueurs internes aux balises cardinales, nous avons
décidé de segmenter les zones de couleur jaune. Pour cela, nous convertissons
l'image en HSV et procédons à un seuillage. La composante H nous permet
d'isoler les zones de couleur jaune. En appliquant un seuillage sur la
composante S nous limitons la détections de bancs de sables qui ont une couleur
proche de celle des balises mais une saturation plus faible.

\subsubsection{Nettoyage}
Nous effectuons ensuite évidemment un nettoyage par ouvertures morphologiques
pour éliminer les petites zones de bruit et les zones touchant les bords de
l'image.

\subsection{Seuillage du gradient sur la composante bleu}

Nous voulons ici obtenir un masque de l'image contenant les balises cardinales
pour pouvoir y appliquer une reconstruction par dilatation.

\subsubsection{Calcul du gradient}
Afin de détecter les contours des balises, nous avons décidé de calculer un
gradient sur la composante bleue de l'image. L'intérêt de cette méthode est de
mettre en évidence les contours des balises qui sont noire et jaune et donc une
très faible composante bleue ce qui crée un fort contraste avec l'arrière plan
généralement bleu.

Nous appliquons un seuillage sur le gradient obtenu pour obtenir le masque
voulu.

\subsubsection{Troncature du gradient}
Afin de supprimer un maximum les grandes zones de plages en fond et de limiter
un maximum les grands objets en arrière plan de nos balises, nous avons décidé
de limiter le calcul du gradient à une zone de l'image délimitée par le masque
de couleur jaune obtenu précédemment. Cette zone est délimitée par les
abscisses minimales et maximales du masque jaune avec une certaine marge pour
ne pas être trop proche des bords de la balise.

\subsection{Reconstruction par dilatation}

Ici nous allons utiliser les marqueurs obtenus précédemment pour effectuer une
reconstruction par dilatation sur le gradient de la composante bleue. Cette
opération nous permet de segmenter les balises cardinales.

Malheureusement, ici nous détectons tout objet contenant une zone jaune. Des
petits objets tels que des bouées ou des bateaux peuvent être détectés. Nous
allons donc devoir effectuer une étape de post-traitement pour ne garder que
les balises cardinales.

\subsection{Suppression des objets non pertinents}

Nous allons maintenant réappliquer la méthode précédente (reconstruction par
dilatation) pour ne garder que les balises cardinales.

Nous calculons donc ici un nouveau masque qui servira de marqueur à la future
reconstruction par dilatation.

\subsubsection{Détection des zones jaunes}
Nous allons ici réappliquer la méthode de seuillage sur la composante jaune de
l'image pour obtenir les zones jaunes dans les balises. Nous appliquons donc
notre calcul des zones jaunes sur l'image masquée par le dernier masque obtenu.

\subsubsection{Suppression des objets trop petits}
Nous allons ici supprimer les objets trop petits pour être des balises
cardinales. Pour cela nous allons calculer l'aire de chaque objet présent et
les filtrer en fonction de leur aire en proportion de l'aire du plus grand
objet.

\subsubsection{Suppression des objets lointains}
Après plusieurs tests, nous avons remarqué que les balises cardinales sont
effectivement trouvées mais qu'il arrive ici que certaines bouées ou objets
assez grand soient détectés. Nous avons donc décidé de supprimer les objets
trop éloignés en abscisse du plus grand objet détecté que nous considérons
comme la partie jaune de la balise cardinale.

\subsubsection{Reconstruction par dilatation}

Nous appliquons donc maintenant une reconstruction par dilatation sur le
premier masque reconstruit en utilisant le masque jaune juste nettoyé.

\subsection{Amélioration de la détection des triangles}

Les tests nous ont montré que la détection des triangles n'était pas parfaite.
En effet, les triangles sont souvent ratés par la détection car ils ne forment
pas une composante connexe avec les balises sur le gradient calculé.

Pour pallier à ce problème, nous avons décidé de faire un dilatation vers le
haut de la balise détectée et de refaire une reconstruction par dilatation sur
le gradient de la composante bleue afin de détecter les triangles
potentiellement oubliés précédemment.

Nous utilisons donc le masque final de l'étape 4 dilaté vers le haut comme
marqueur pour la reconstruction par dilatation.

Les resultats de cette méthode sont relativement satisfaisant et nous
permettent de détecter plus de triangles que précédemment.

\section{Classification par couleur}

Nous avons maintenant un masque de l'image qui ne contient que la balise
classifier. L'objectif de cette partie est de classifier la balise en fonction
de sa couleur. Pour cela nous allons analyser la position des zones jaunes de
la balise.

\subsection{Boîte englobante de la balise}

Nous allons ici calculer la boîte englobante de la balise pour pouvoir
déterminer la position des zones jaunes par rapport à la balise.

En considérant que le masque ne comprend que la balise, nous pouvons calculer
la boîte englobante de la balise en utilisant la boîte englobante du masque.

La boîte englobante d'intérêt est la boîte englobante de la balise sans les
triangles. Pour isoler cette zone de la balise, nous avons décidé de supprimer
les 22 plus hauts pourcents de la boîte englobante de la balise.

\subsection{Détection des boîtes englobantes des zones jaunes}

Afin de calculer la position des zones jaunes par rapport à la balise, nous
avons besoin de leurs boîtes englobantes.

Nous faisons donc une détection des zones jaunes par seuillage en HSV et
fusionnons les bounding boxes des zones jaunes trop proches ou se chevauchant.

Une étape de nettoyage est également effectuée pour supprimer les zones jaunes
trop petites. qui pourrait correspondre à des bruits ou à des ombres sur le
pied de la balise par exemple.

\subsection{Classification}

Nous allons maintenant classifier la balise en fonction de la position des
zones jaunes par rapport à la balise.

Un premier filtre est effectué dans le cas ou 2 zones jaunes sont détectés car
seules les balises cardinales Ouest ont 2 zones jaunes.

Nous calculons ensuite si la zone jaune restante est présente dans la partie
haute, médiane ou basse de la boîte englobante de la balise. Cela nous permet
de déterminer si la balise est une balise cardinale Nord, Est ou Sud.

\section{Classification par triangles}

Dans cette partie, nous repartons du masque binaire de l’image, qui ne contient
que la balise classifier. Ce masque est une image en noir et blanc. L’objectif
de cette partie est de classifier la balise en fonction de l’orientation des
triangles. Pour cela nous allons analyser la position des triangles de la
balise.

\subsection{Boîte englobante des triangles}

En partant de la Bounding Box ne contenant en théorie que la balise, on prend
la partie supérieure contenant les triangles. En effet, ils représentent
environ 22\% de sa hauteur, ce qui facilite l’extraction de cette partie de
l’image. On obtient alors uniquement les 2 triangles de la balise étudiée,
l’inverse de la partie précédente qui se concentrait sur le reste de la balise
sans les triangles.

\subsection{Nettoyage des triangles}

Pour simplifier l’étape suivante et garantir des triangles bien définis, un
léger nettoyage est effectué. Cela consiste en une opération morphologique
d’ouverture avec un disque comme élément structurant. Ce traitement permet
d’éliminer les éventuels bruits autour des triangles, ainsi que la barre qui
les supporte.

\subsection{Classification}

Enfin, l’image contenant les quatre triangles est divisée en quatre sous-images
de taille égale, correspondant chacune à un demi-triangle. Pour identifier la
balise, nous comptons le nombre de pixels blancs dans chaque demi-triangle.
Selon la distribution des pixels, nous déterminons le type de balise.

Exemple : Si la première image a plus de pixels que la deuxième alors le
premier triangle correspond à une balise Sud ou Ouest. De même, si la troisième
image en a plus que la quatrième alors c’est une balise Sud. Ce qui nous donne
un résultat pour la classification par triangles.

\section{Double Validation}

Afin de d'améliorer la précision de notre algorithme, nous avons décidé de
mettre en place une double validation. L'idée est de vérifier que la
classification par couleur et par forme sont cohérentes. Pour cela, nous allons
comparer les résultats obtenus par les deux méthodes et ne conserver que les
balises qui sont classées de la même manière par les deux méthodes.

\subsection{Paramètre de cohérence}
Nous créons donc un paramètre de confiance nommé cohérence (consistency) qui
est un booléen TRUE quand les 2 algorithmes sont cohérents et FALSE sinon.

La direction de navigation conseillée, explicitée dans la partie suivante, est
calculée seulement dans les cas cohérents.

\section{Recherche d'une direction de navigation}

La recherche de la direction de navigation est simple et ne se fait que par
simple calcul de la position des balises par rapport à la position du bateau.

Nous affichons la direction de navigation conseillée en fonction de la position
des balises cardinales détectées directement sur l'image demandée.

\section{Analyse des résultats}

Les résultats suivants sont les résultats de classification des balises
cardinales sur les images de test. Ces résultats et statistiques sont calculés
sur le jeu de données de test de l'énoncé. Ci-dessous, le tableau des
résultats.

\begin{table}[h!]
    \centering
    \resizebox{\textwidth}{!}{%
        \begin{tabular}{|c|p{4cm}|c|c|c|}
            \hline
            \rowcolor{gray!60}
            \textbf{Image}       & \textbf{Segmentation Result} & \textbf{Color Classification} & \textbf{Triangle Classification} & \textbf{Consistency} \\ \hline
            
            \textbf{60dsP35.jpg} & Background kept              & West                          & North                            & FALSE                \\ \hline
            \rowcolor{gray!10}
            \textbf{7yVN7Wh.jpg} &                              & East                          & South                            & FALSE                \\ \hline
            \rowcolor{gray!0}
            \textbf{86LjH8X.jpg} & Water under kept             & South                         & South                            & TRUE                 \\ \hline
            \rowcolor{gray!10}
            \textbf{8K2kvSA.jpg} & Water under kept             & South                         & South                            & TRUE                 \\ \hline
            \rowcolor{gray!0}
            \textbf{A6hVLZ7.jpg} &                              & East                          & North                            & FALSE                \\ \hline
            \rowcolor{gray!10}
            \textbf{J3JGeTY.jpg} & Lack 1 triangle              & South                         & West                             & FALSE                \\ \hline
            \rowcolor{gray!0}
            \textbf{O1XtIg5.jpg} & Water under kept             & South                         & South                             & TRUE                \\ \hline
            \rowcolor{gray!10}
            \textbf{YU750z5.jpg} & A bit of land behind         & South                         & South                            & TRUE                 \\ \hline
            \rowcolor{gray!0}
            \textbf{h3YP60d.jpg} & Water around kept            & West                          & North                            & FALSE                \\ \hline
            \rowcolor{gray!10}
            \textbf{i4R1U6v.jpg} & Half, lack triangles         & East                          & South                            & FALSE                \\ \hline
            \rowcolor{gray!40}
            \textbf{South}       &                              & 50\%                          & 60\%                             & 40\%                 \\ \hline
            \rowcolor{gray!0}
            \textbf{3DTzEGh.jpg} & Water around kept            & East                          & South                            & FALSE                \\ \hline
            \rowcolor{gray!10}
            \textbf{7F08T3Y.jpg} &                              & East                          & East                             & TRUE                 \\ \hline
            \rowcolor{gray!0}
            \textbf{8b3J5Me.jpg} & Lack triangles               & East                          & North                            & FALSE                \\ \hline
            \rowcolor{gray!10}
            \textbf{G3amY5r.jpg} & Triangles blurred            & East                          & East                             & TRUE                 \\ \hline
            \rowcolor{gray!0}
            \textbf{MI8n706.jpg} & Lack 1 triangle              & East                          & West                             & FALSE                \\ \hline
            \rowcolor{gray!10}
            \textbf{QO80o8u.jpg} &                              & East                          & East                             & TRUE                 \\ \hline
            \rowcolor{gray!0}
            \textbf{SHmk487.jpg} &                              & East                          & East                             & TRUE                 \\ \hline
            \rowcolor{gray!10}
            \textbf{UJNcds9.jpg} &                              & East                          & East                             & TRUE                 \\ \hline
            \rowcolor{gray!0}
            \textbf{d5KA2X6.jpg} & A bit of land behind         & East                          & East                             & TRUE                 \\ \hline
            \rowcolor{gray!10}
            \textbf{tA2n98e.jpg} & Cut in half horizontally     & North                         & East                             & FALSE                \\ \hline
            \rowcolor{gray!40}
            \textbf{East}        &                              & 90\%                          & 70\%                             & 60\%                 \\ \hline
            \rowcolor{gray!0}
            \textbf{3bFnQfx.jpg} &                              & North                         & North                            & TRUE                 \\ \hline
            \rowcolor{gray!10}
            \textbf{3zmY4P1.jpg} & Water under kept             & East                          & North                            & FALSE                \\ \hline
            \rowcolor{gray!0}
            \textbf{5I878xn.jpg} & Lack 1 triangle              & East                          & East                             & TRUE                 \\ \hline
            \rowcolor{gray!10}
            \textbf{5epK341.jpg} & Holes in triangles           & North                         & North                            & TRUE                 \\ \hline
            \rowcolor{gray!0}
            \textbf{863P5k3.jpg} & Land behind                  & North                         & North                            & TRUE                 \\ \hline
            \rowcolor{gray!10}
            \textbf{Pn1PEIM.jpg} &                              & North                         & North                            & TRUE                 \\ \hline
            \rowcolor{gray!0}
            \textbf{TgfIcDT.jpg} &                              & North                         & North                            & TRUE                 \\ \hline
            \rowcolor{gray!10}
            \textbf{Y79Jt8c.jpg} & Land behind                  & North                         & North                            & TRUE                 \\ \hline
            \rowcolor{gray!0}
            \textbf{bsO98Tk.jpg} & A bit of land behind         & East                          & North                            & FALSE                \\ \hline
            \rowcolor{gray!10}
            \textbf{fVrzdzC.jpg} & Triangles cut                & North                         & West                             & FALSE                \\ \hline
            \rowcolor{gray!0}
            \textbf{oeSph7h.jpg} & A bit of water under kept    & North                         & North                            & TRUE                 \\ \hline
            \rowcolor{gray!40}
            \textbf{North}       &                              & 72,73\%                       & 81,82\%                          & 72,73\%              \\ \hline
            \rowcolor{gray!0}
            \textbf{01Tn6V3.jpg} &                              & West                          & West                             & TRUE                 \\ \hline
            \rowcolor{gray!10}
            \textbf{08hYdUe.jpg} &                              & West                          & West                             & TRUE                 \\ \hline
            \rowcolor{gray!0}
            \textbf{63JKJVG.jpg} & Water under kept             & West                          & West                             & TRUE                 \\ \hline
            \rowcolor{gray!10}
            \textbf{65MsTb3.jpg} &                              & West                          & West                             & TRUE                 \\ \hline
            \rowcolor{gray!0}
            \textbf{8KzVm58.jpg} & Lack triangles               & South                         & West                             & FALSE                \\ \hline
            \rowcolor{gray!10}
            \textbf{Ao1UT0B.jpg} &                              & West                          & West                             & TRUE                 \\ \hline
            \rowcolor{gray!0}
            \textbf{IY2fAd3.jpg} &                              & West                          & West                             & TRUE                 \\ \hline
            \rowcolor{gray!10}
            \textbf{QI7MH37.jpg} & Water under kept             & West                          & West                             & TRUE                 \\ \hline
            \rowcolor{gray!0}
            \textbf{XP96z9K.jpg} & Lack triangles               & West                          & East                             & FALSE                \\ \hline
            \rowcolor{gray!10}
            \textbf{mtjUaK0.jpg} & Lack triangles               & West                          & North                            & FALSE                \\ \hline
            \rowcolor{gray!40}
            \textbf{West}        &                              & 90\%                          & 80\%                             & 70\%                 \\ \hline
            \rowcolor{gray!40}
            \textbf{Total}       &                              & \textbf{75,61\%}              & \textbf{73,17\%}                 & \textbf{60,97\%}     \\ \hline
        \end{tabular}%}
    }
    \caption{Résultats de classification des balises cardinales}\label{tab:segmentation_results}
\end{table}

\subsection{Segmentation}

On remarque que la segmentation des balises n'est pas parfaite. En effet, il
manque parfois des triangles ou des parties de balises. Cela peut être dû à la
qualité de l'image, à la présence d'ombres ou à la présence d'autres objets
dans l'image qui peuvent interférer avec la détection des balises.

Les problèmes les plus courants sont dus au fort gradient dans les vagues ou à
des objets en fond qui sont confondus avec la balise dans le gradient de
l'image.

On perd également parfois des triangles malgré notre méthode d'agrandissement
puis de reconstruction par dilatation.

\subsection{Classification par couleur}

La classification par couleur est relativement précise. Nous remarquons que
dans les cas ou la segmentation est bonne ou que l'on detecte un objet en
arrière plan, la classification par couleur est correcte. Cette méthode semble
donc globalement efficace pour classer les balises cardinales.

\subsection{Classification par triangles}

La classification par triangles semble moins efficace que la classification par
couleurs au vu des résultats généraux mais elle est très fortement dépendante
de la segmentation des balises. En effet, si la segmentation est mauvaise, il
est difficile de détecter les triangles.

On remarque en effet que la plupart des erreurs de cette méthode sont dues à
une mauvaise segmentation des balises plutôt qu'à une mauvaise classification
des triangles.

\section{Conclusion}

En conclusion, notre algorithme est relativement efficace pour classifier les
balises cardinales. La classification par couleur est plus précise que la
classification par triangle car plus robuste aux mauvaises segmentations. La
double validation permet d'améliorer la précision de notre algorithme.
Cependant, la segmentation des balises reste perfectible. Il serait intéressant
d'explorer d'autres méthodes de segmentation pour améliorer les résultats. Il
serait également intéressant de tester notre algorithme sur un plus grand jeu
de données pour évaluer sa robustesse.

Il pourrait également être intéressant d'explorer la piste d'une detection des
blocs noirs pour améliorer les résultats de la segmentation par couleur dans
les cas ou les balises ont une zone jaune grande en comparaison de la zone
noire.

La méthode par détection de triangles pourrait également être améliorée en
utilisant des méthodes de détection des contours plus précises et en appliquant
peut être une seconde segmentation lors de cette étape.

\end{document}